
\usetheme[light,darktitle,framenumber,totalframenumber]{UniversiteitAntwerpen}

\usepackage{tabulary}
\usepackage{changepage}
\usepackage{tikz}
\usepackage{booktabs}
\usepackage{listings}
\usepackage{lstlinebgrd}
\usepackage{tcolorbox}
\usepackage{ifthen}
\usepackage[absolute,overlay]{textpos}
%\usepackage{enumitem}


% Reset itemize stuff after enumitem
%\setitemize{label=\usebeamerfont*{itemize item}%
%\usebeamercolor[fg]{itemize item}
%\usebeamertemplate{itemize item}}


\usetikzlibrary{calc}


\usepackage[style=authortitle,backend=bibtex]{biblatex}
\addbibresource{bibfile.bib}

\usepackage{listings}
\lstset{
	tabsize=4,
	gobble=12,
	escapechar=!
}
\fboxsep1pt

\addbibresource{bibfile.bib}

% No navigation symbols
\beamertemplatenavigationsymbolsempty

% add space between footer and footnotes
\addtobeamertemplate{footline}{
\raisebox{0.5cm}{
}
}{
}

%\AtBeginSection[]{
%\begin{frame}
%%	\frametitle{\insertsection}
%	\frametitle{Outline}
%	\tableofcontents[currentsection]
%\end{frame}
%}

\newcommand{\includegraphicsframe}[1]{
		\bgroup
		\usebackgroundtemplate{\includegraphics[keepaspectratio, width=\paperwidth, height=\paperheight]{#1}}%
		\begin{frame}[plain, noframenumbering]
		\end{frame}
		\egroup
}

\newcommand{\tikzmark}[1]{\tikz[overlay,remember picture] \node (#1) {};}
\newcommand{\DrawBox}[3][]{%
	\tikz[overlay,remember picture]{
		\draw[black, #1]
		($(#2)+(-0.5em,2.0ex)$) rectangle
		($(#3)+(0.75em,-0.75ex)$);}
}

\newcommand<>{\highlight}[1]{\textcolor#2{uared}{\textbf{#1}}}
\newcommand{\raisecenter}[1]{\raisebox{-0.5\height}{#1}}

\newcommand{\todo}[1]{
	\par 
	\fcolorbox{red}{white}{
		\textcolor{red}{TODO: #1} %\marginpar{#1}
	}
	\par
}

%%%%%%%%%%%%%%%%%%%%%%%%%%%%%%%%%%%%%%%%%%%%%%%%%%%%%%%%%%%%%%%%%%%%%
%  listings highlighting (https://tex.stackexchange.com/a/85628)	%
%%%%%%%%%%%%%%%%%%%%%%%%%%%%%%%%%%%%%%%%%%%%%%%%%%%%%%%%%%%%%%%%%%%%%
\makeatletter
%%%%%%%%%%%%%%%%%%%%%%%%%%%%%%%%%%%%%%%%%%%%%%%%%%%%%%%%%%%%%%%%%%%%%%%%%%%%%%
%
% \btIfInRange{number}{range list}{TRUE}{FALSE}
%
% Test in int number <number> is element of a (comma separated) list of ranges
% (such as: {1,3-5,7,10-12,14}) and processes <TRUE> or <FALSE> respectively

\newcount\bt@rangea
\newcount\bt@rangeb

\newcommand\btIfInRange[2]{%
	\global\let\bt@inrange\@secondoftwo%
	\edef\bt@rangelist{#2}%
	\foreach \range in \bt@rangelist {%
		\afterassignment\bt@getrangeb%
		\bt@rangea=0\range\relax%
		\pgfmathtruncatemacro\result{ ( #1 >= \bt@rangea) && (#1 <= \bt@rangeb) }%
		\ifnum\result=1\relax%
		\breakforeach%
		\global\let\bt@inrange\@firstoftwo%
		\fi%
	}%
	\bt@inrange%
}
\newcommand\bt@getrangeb{%
	\@ifnextchar\relax%
	{\bt@rangeb=\bt@rangea}%
	{\@getrangeb}%
}
\def\@getrangeb-#1\relax{%
	\ifx\relax#1\relax%
	\bt@rangeb=100000%   \maxdimen is too large for pgfmath
	\else%
	\bt@rangeb=#1\relax%
	\fi%
}

%%%%%%%%%%%%%%%%%%%%%%%%%%%%%%%%%%%%%%%%%%%%%%%%%%%%%%%%%%%%%%%%%%%%%%%%%%%%%%
%
% \btLstHL<overlay spec>{range list}
%
% 
\newcommand<>{\btLstHL}[1]{%
	\only#2{\btIfInRange{\value{lstnumber}}{#1}{\color{orange!30}\def\lst@linebgrdcmd{\color@block}}{\def\lst@linebgrdcmd####1####2####3{}}}%
}%
\makeatother

